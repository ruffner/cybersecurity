\documentclass[10pt,twocolumn,letterpaper]{article}

\usepackage{wifs}
\usepackage{times}
\usepackage{epsfig}
\usepackage{graphicx}
\usepackage{amsmath}
\usepackage{amssymb}

% Include other packages here, before hyperref.

% If you comment hyperref and then uncomment it, you should delete
% egpaper.aux before re-running latex.  (Or just hit 'q' on the first latex
% run, let it finish, and you should be clear).
\usepackage[pagebackref=true,breaklinks=true,letterpaper=true,colorlinks,bookmarks=false]{hyperref}

\wifsfinalcopy % *** Uncomment this line for the final submission

\def\wifsPaperID{****} % *** Enter the WIFS Paper ID here
\def\httilde{\mbox{\tt\raisebox{-.5ex}{\symbol{126}}}}

% Pages are numbered in submission mode, and unnumbered in camera-ready
\ifwifsfinal\pagestyle{empty}\fi
\begin{document}

%%%%%%%%% TITLE
\title{ Cyberphysical Attacks on Modern Vehicle Control Systems}

\author{Matt Ruffner\\
University of Kentucky\\
Lexington, KY 40506-0046\\
{\tt\small matthew.ruffner@uky.edu}}
% For a paper whose authors are all at the same institution,
% omit the following lines up until the closing ``}''.
% Additional authors and addresses can be added with ``\and'',
% just like the second author.
% To save space, use either the email address or home page, not both


\maketitle
%\thispagestyle{empty}

%%%%%%%%% ABSTRACT
\begin{abstract}
  Modern vehicles have an increasing number of electronic control units, with some having cellular modems and other internet-facing gateways. This demands new means to protect these systems from cyberphysical attacks. These systems present a diverse array of susceptible subsystems, from remote entry immobilization, throttle control, and data such as speed and coordinates from a navigation system. This paper examines the established means by which cars are a target for cyberphysical attacks. As an extension, a discussion on ways in which the even-more-connected cars of the future may be guarded from such attacks is also included. 
\end{abstract}

%%%%%%%%% BODY TEXT
\section{Introduction}

This is a citation~\cite{KoscherKarl2010ESAo}.

This is the one on attack surfaces~\cite{Checkoway:2011:CEA:2028067.2028073}.

classifying robotic anomolies~\cite{BezemskijAnatolij2016BADo}

securing the CAN bus~\cite{CabertoEddie2017AMoS}.


{\small
\bibliographystyle{ieee}
\bibliography{refs.bib}
}

\end{document}
